\part{Grupo Fundamental}

\begin{quote}{H. Poincaré (Acta Mathematica, vol. 38 (1921) pag. 101.)}
\textit{``Quant à moi, toutes les voies diverses où je m`étais engagé
successivement me conduisaient à l`Analysis Situs. J`avais besoin
des donnés de cette Science pour poursuivre mes études sur les
courbes définies par les équations différentielles et pour les étendre
aux équations différentielles d`ordre superieur et, en particulier,
à celles du problème de trois corps. J`en avais besoin pour l` étude
des fonctions non uniformes de deux variables. J`en avais besoin
pour l` étude au developpement de la fonction perturbatrice. Enfin,
j`entrevoyais dans l`Analysis Situs un moyen d`aborder un problème
important de la théorie des groupes, la recherche des groupes 
discrets ou des groupes finis contenus dans un groupe continu donné.''}
\end{quote}

Henri Poincaré (1854-1912), extraordinário matemático francês, era
tido como ``o último universalista'', isto é, contribuidor para o
progresso de todos os ramos importantes da Matemática. A ele se deve a
noção de grupo fundamental e a criação da teoria da homologia,
conceitos basiliares da Topologia.

A citação acima, escrita em sua auto-biografia científica (''Notice
sur les travaux scientifiques de Henri Poincaré''), foi publicada pela
primeira vez nove anos após sua morte, num número especial da revista
sueca ``Acta Mathematica'', a ele dedicado.
