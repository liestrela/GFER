\chapter{Homotopia}

\textbf{Em todo este livro, o símbolo $I$ signfiica o intervalo
compacto $[0,1]$}.

Neste capítulo serão introduzidas noções básicas sobre homotopia, as
quais vão ser utilizadas em todo o livro. Homotopia é, na realidade, a
idéia mais importante da Topologia Algébrica e o grupo fundamental,
que estudaremos aqui, é provavelmente o invariante algébrico mais
simples associado a essa idéia. O grupo fundamnetal será apresentado
no capítulo seguinte. Neste, trataremos de fatos gerais a respeito da
homotopia, procurando ilustrá-los com aplicações e exemplos
elementares, mostrando, em particular, a estreita relação entre
homotopia e o problema de estender continuamente a todo o espaço uma
aplicação contínua definida num subconjunto fechado desse espaço.

\section{Aplicações homotópicas}

Sejam $X$, $Y$ espaços topológicos. Duas aplicações coninuas $f, g
\colon X \to Y$ dizem-se \textit{homotópicas} quando existe uma
aplicação contínua

\begin{equation*}
	H \colon X \times I \to Y
\end{equation*}

tal que $H(x, 0) = f(x)$ e $H(x, 1) = g(x)$ para todo $x \in X$. A
aplicação $H$ chama-se então uma \textit{homotopia} entre $f$ e $g$.
Escreve-se, neste caso, $H \colon f \simeq g$, ou simplesmente $f
\simeq g$.

Dada a homotopia $H \colon f \simeq g$, consideremos, para cada $t \in
I$, a aplicação contínua $H_{t} \colon X \to Y$, definida por
$H_t(x) = H(x, t)$. Dar a homotopia $H$ equivale a definir uma
``família contínua a um parâmetro'' $(H_t)_{t \in I}$ de aplicações
de $X$ em $Y$. A ``continuidade'' da família significa, neste caso,
que $(x, t) \mapsto H_t(x)$ é um aplicação contínua. Temos $H_0 = f$
e $H_1 = g$, de modo que a família $(H_t)_{t \in I}$ começa com $f$ e
termina com $g$.

Intuitivamente, o parâmetro $t$ pode ser imaginado como sendo o tempo.
A homotopia é então pensada com um processo de deformação coninua da
aplicação $f$. Tal deformação tem lugar durante uma unidade de tempo.
No instante $t=0$ temos $f$; para $t=1$ temos g. Nos instantes
intermediários, $0<t<1$, as aplicaoes $H_t$ fornecem os estágios
intermediários da deformação.

\textbf{Observação.} Podemos considerar o espaço topológico $C(X;Y)$,
formado pelas aplicações contínuas de $X$ em $Y$, com a topologia
compacto-aberta. (No caso de $Y$ ser metrizável, esta é a topologia da
convergência uniforme nas partes compactas de $X$.) A cada aplicação
$H \colon X \times I \to Y$ corresponde um caminho em $C(X;Y)$, isto
é, uma aplicação $\widetilde{H} \colon I \to C(X;Y)$ definida por
$\widetilde{H}(t) = H_t, H_t(x) = H(t,x)$. Quando $X$ é um espaço de
Hausdorff localmente compacto, ou é metrizável, então $H$ é contínua
se, e somente se $\widetilde{H}$ o for. Assim, para $X$ metrizável ou
localmente compacto de Hausdorff, existe uma bijeção natural entre as
homotopias $H \colon X \times I \to Y$ e os caminhos $\widetilde{H}
\colon I \to C(X;Y)$. Se $H$ é uma homotopia entre $f$ e $g$ então o
caminho $\widetilde{H}$ começa em $f,g \colon X \to Y$, com $X$
localmente compacto de Hausdorff ou metrizável são homotópicas se, e
somente se, $f$ e $g$ pertencem à mesma componente conexa por caminhos 
no espaço $C(X;Y)$. Para a topologia compacto-aberta, ver Bredon[1],
página 437.

\textbf{Exemplo 1.} Duas aplicações constantes $f, g \colon X \to Y,
f(x) = p, g(x) = q$, são homotópicas se, e somente se, $p$ e $q$
pertencem à mesma componente conexa por caminhos do espaço $Y$. Com
efeito, se existe um caminho $a \colon I \to Y$ com $a(0) = p$ e $a(1)
= q$, definimos uma homotopia $H \colon X \times I \to Y$ entre $f$ e
$g$ pondo $H(x, t) = a(t)$, para todo $(x, t) \in X \times I$.
Reciprocamente, se $H$ é uma homotopia entre as aplicações constantes
$f(x) = p$ e $g(x) = q$, fixando arbitrariamente $x_0 \in X$,
obteremos um caminho $a \colon I \in Y$ ligando $p$ a $q$ pondo $a(t)
= H(x_0, t)$.
